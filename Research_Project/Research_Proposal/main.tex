\documentclass[12pt,a4paper]{article}

% 1. Encoding and Core Language
\usepackage[T1]{fontenc}
\usepackage[utf8]{inputenc}
\usepackage[english]{babel}
\usepackage{csquotes}
\usepackage{xcolor} % Loaded early to avoid conflicts with TikZ

% 2. Math and Physics (Load before table packages)
\usepackage{amsmath}
\allowdisplaybreaks
\usepackage{amssymb}
\usepackage{amsfonts}

% 3. Graphics and TikZ
\usepackage{graphicx}
\usepackage{caption}
\usepackage{tikz}
\usetikzlibrary{positioning, arrows.meta, shapes.geometric, shadows, patterns}

% 4. Bibliography (Must come before hyperref)
\usepackage[
    backend=biber, 
    style=ieee,
    maxnames=3,
    giveninits=true,
    sorting=none,
    doi=true,
    isbn=false,
    sortcites,
]{biblatex}

% 5. Tables (Specific order: array -> booktabs -> tabularx -> longtable -> xltabular)
\usepackage{array}
\usepackage{booktabs}
\usepackage{multirow}
\usepackage{colortbl}
\usepackage{tabularx}
\usepackage{longtable}
\usepackage{xltabular}
\usepackage{pdflscape}
\usepackage{float}

% 6. Layout
\usepackage{geometry}
\geometry{left=2cm,right=2cm,top=2cm,bottom=2cm}
\setlength{\parindent}{0pt}

% 7. THE FINALE: Referencing (Order is mandatory!)
% \usepackage{url}
% \usepackage[
%     unicode=true, 
%     hidelinks, 
%     colorlinks=true, 
%     allcolors=blue,
%     bookmarks=true,         % Enables bookmarks
%     bookmarksnumbered=true, % Shows the section numbers in the navigation
%     bookmarksopen=true      % Opens the sidebar by default when the PDF is opened
% ]{hyperref}
% \usepackage[capitalize, noabbrev]{cleveref}
\usepackage[unicode=true]{hyperref} % enables use of metadata for pdfs and hyperlinks within a document
\hypersetup{%colorlinks=true
% activate this part for the boxes 
colorlinks=false, %flag for prints
pdfborder={0 0 1}, % thickness of box
linkbordercolor={1 0 0},
% activate this part to not have the boxes
% hidelinks,  % this option would hide links for the print version of your thesis
% linkcolor=red!35!black,    %definition of the link color
% citecolor=green!35!black,  %definition of the cite color
% urlcolor=magenta!35!black, %definition of the url color
pdfauthor=Simon Kraft, % Optional: Specify the author of the pdf
pdftitle=Assignment 1   % Optional: Specify the title within the pdf
} 
\usepackage[capitalize,noabbrev]{cleveref}
% \usepackage{bookmark}

\addbibresource{references.bib}

\usepackage[utf8]{inputenc}
\usepackage{array} % For better table row heights


% \addbibresource{references.bib} % Note the .bib extension is required here

% --- Document Metadata ---
\title{
    % \vspace{5em}
    \large \textbf{CPSC 644 - Computer Networks} \\

    \huge \textbf{Research Proposal} \\
}

\author{
    \large Submitted by \\[0.2cm]
    \renewcommand{\arraystretch}{1.5}
    \begin{tabular}{|c|c|}
        \hline
        \textbf{Student ID} & \textbf{Name} \\
        \hline
        230171256 & Simon Kraft \\
        \hline
    \end{tabular}
}

\date{\today}

\begin{document}

\maketitle
\newpage

\section{Topic Choice}

For my research paper I have chosen the topic ''AI-Driven Network Management and Self-Healing Networks''.

\section{Description}

Defining this research topic requires defining two closely related principles and also identifying their differences. First, ''AI-Driven Network Management'', often referred to as ''AI Networking'', is a generic term that describes the integration of Artificial Intelligence (AI) and Machine Learning (ML) into the area of computer networks with the goal of improving efficiency and effectiveness. This includes areas such as dynamic resource allocation, fault detection and analysis, and security enhancement~\cite{IntegrationofAIinNetworkManagement_2024}. Second, ''Self-Healing Networks'' represent networks that can autonomously detect, diagnose, and resolve network failures in real time without the need for human intervention. This is often achieved by using AI or Machine Learning (ML) in a closed-loop system to continuously monitor performance and quickly detect disruptions or irregularities and responding by dynamically rerouting traffic, adjusting configurations, or isolating problematic components~\cite{Stone2025}. \\

Both principles differ in their objective. While AI Networking tries to integrate \textit{intelligence} into the network to make it more efficient and secure, Self-Healing Networks aim to make networks self-sustaining and more resilient. For this reason, Self-Healing Networks could be seen as a subset of AI Networking. 

\section{Motivation}

Modern computer networks are constantly growing and becoming more complex, exceeding the capabilities of human oversight by far. Therefore, rule-based and manual methods for network monitoring are insufficient for the scale, complexity, and sophistication of today's networks, that connect thousands of devices simultaneously~\cite{China2025}. According to a report from Dynatrace, a company that develops AI solutions for observability, the average multicloud environment spans 12 different services and platforms~\cite{Dynatrace2024}. Furthermore, their report finds that more than 80\% of technology leaders say that the effort their teams spent on maintaining monitoring tools and preparing data for analysis steals time form innovation. \\

These circumstances highlight the need for AI Networking and Self-Healing Networks to analyze large amounts of data in real-time and identifying and address potential network issues before they grow into larger problems~\cite{Tailscale2026}. Integrating these mechanisms reduces costs by optimizing network resources, decreasing operational downtime, and automating routine tasks to free up IT staff for higher-level network management jobs~\cite{China2025}. \\

However, integrating AI into the area of network management poses significant challenges that shouldn't be underestimated. For instance, data privacy concerns, high investment costs for infrastructure building and skilled personnel, as well as the risk of handing over the control for critical enterprise infrastructure to an AI which does not necessarily chooses the correct decision in each situation is dangerous~\cite{Tailscale2026}.


\section{Proposed Paper Outline}

\subsubsection*{Abstract}

\subsubsection*{Introduction}

\subsubsection*{Literature Survey}

\subsubsection*{Real-World Usage}

\subsubsection*{Emerging Trends}

\subsubsection*{Conclusion}

\subsubsection*{Bibliography}


\section{Bibliography}

\newpage

\printbibliography



\newpage

\section*{Mark Breakdown}

Your proposal should be a detailed outline of the research paper you will produce. The proposal should include for the selected research topic the sections: description, motivation, outline, and bibliography. The main body (not including the bibliography, cover page) must not exceed two pages in length (single spaced, single column, 12pt font)

\begin{itemize}
    \item \textbf{Topic Choice (2.5 marks):} Evaluation based on relevance, innovation, and level of difficulty.
    \item \textbf{Description (5 marks):} A clear definition of what the research topic is.
    \item \textbf{Motivation (2.5 marks):} An explanation of why the research topic is interesting or significant.
    \item \textbf{Outline (15 marks):} A clear, focused outline stating precisely what the proposed research will cover. 
    \begin{itemize}
        \item Must include appropriate citations.
        \item Each section/subsection requires a short summary (3--4 sentences) describing the focus of the final research paper in that section.
    \end{itemize}
    \item \textbf{Bibliography (15 marks):} A well-researched list of sources.
    \begin{itemize}
        \item Minimum of 8 sources total (at least 3 recent reviewed sources per team member).
        \item At least 75\% must be reputable sources (e.g., peer-reviewed journals, conferences, established community sources).
        \item Textbook usage is permitted.
        \item Any reputable bibliography style is acceptable (IEEE or ACM recommended) if used consistently.
        \item \textit{Note:} Ensure the paper format remains single column.
    \end{itemize}
\end{itemize}

\section*{Topic Description}

AI-Driven Network Management and Self-Healing Networks \\

Explores the integration of machine learning for automated fault detection, prediction, and real-time optimization in large-scale networks. \\

\textit{Motivation: Reduces downtime and operational costs in complex, dynamic
environments.}




\end{document}